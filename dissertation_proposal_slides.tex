\documentclass[aspectratio=169]{beamer}

% Theme and color scheme
\usetheme{Madrid}
\usecolortheme{default}

% Packages
\usepackage{amsmath}
\usepackage{amssymb}
\usepackage{amsthm}
\usepackage{graphicx}
\usepackage{tikz}
\usepackage{booktabs}
\usepackage{natbib}

% Custom commands from your thesis
%%%%% NEW MATH DEFINITIONS %%%%%

\usepackage{amsmath}
\usepackage{amsfonts}
\usepackage{bm}

% Mark sections of captions for referring to divisions of figures
\newcommand{\figleft}{{\em (Left)}}
\newcommand{\figcenter}{{\em (Center)}}
\newcommand{\figright}{{\em (Right)}}
\newcommand{\figtop}{{\em (Top)}}
\newcommand{\figbottom}{{\em (Bottom)}}
\newcommand{\captiona}{{\em (a)}}
\newcommand{\captionb}{{\em (b)}}
\newcommand{\captionc}{{\em (c)}}
\newcommand{\captiond}{{\em (d)}}

% Highlight a newly defined term
\newcommand{\newterm}[1]{{\bf #1}}


% Figure reference, lower-case.
\def\figref#1{figure~\ref{#1}}
% Figure reference, capital. For start of sentence
\def\Figref#1{Figure~\ref{#1}}
\def\twofigref#1#2{figures \ref{#1} and \ref{#2}}
\def\quadfigref#1#2#3#4{figures \ref{#1}, \ref{#2}, \ref{#3} and \ref{#4}}
% Section reference, lower-case.
\def\secref#1{section~\ref{#1}}
% Section reference, capital.
\def\Secref#1{Section~\ref{#1}}
% Reference to two sections.
\def\twosecrefs#1#2{sections \ref{#1} and \ref{#2}}
% Reference to three sections.
\def\secrefs#1#2#3{sections \ref{#1}, \ref{#2} and \ref{#3}}
% Reference to an equation, lower-case.
\def\eqref#1{equation~\ref{#1}}
% Reference to an equation, upper case
\def\Eqref#1{Equation~\ref{#1}}
% A raw reference to an equation---avoid using if possible
\def\plaineqref#1{\ref{#1}}
% Reference to a chapter, lower-case.
\def\chapref#1{chapter~\ref{#1}}
% Reference to an equation, upper case.
\def\Chapref#1{Chapter~\ref{#1}}
% Reference to a range of chapters
\def\rangechapref#1#2{chapters\ref{#1}--\ref{#2}}
% Reference to an algorithm, lower-case.
\def\algref#1{algorithm~\ref{#1}}
% Reference to an algorithm, upper case.
\def\Algref#1{Algorithm~\ref{#1}}
\def\twoalgref#1#2{algorithms \ref{#1} and \ref{#2}}
\def\Twoalgref#1#2{Algorithms \ref{#1} and \ref{#2}}
% Reference to a part, lower case
\def\partref#1{part~\ref{#1}}
% Reference to a part, upper case
\def\Partref#1{Part~\ref{#1}}
\def\twopartref#1#2{parts \ref{#1} and \ref{#2}}

\def\ceil#1{\lceil #1 \rceil}
\def\floor#1{\lfloor #1 \rfloor}
\def\1{\bm{1}}
\newcommand{\train}{\mathcal{D}}
\newcommand{\valid}{\mathcal{D_{\mathrm{valid}}}}
\newcommand{\test}{\mathcal{D_{\mathrm{test}}}}

\def\eps{{\epsilon}}


% Random variables
\def\reta{{\textnormal{$\eta$}}}
\def\ra{{\textnormal{a}}}
\def\rb{{\textnormal{b}}}
\def\rc{{\textnormal{c}}}
\def\rd{{\textnormal{d}}}
\def\re{{\textnormal{e}}}
\def\rf{{\textnormal{f}}}
\def\rg{{\textnormal{g}}}
\def\rh{{\textnormal{h}}}
\def\ri{{\textnormal{i}}}
\def\rj{{\textnormal{j}}}
\def\rk{{\textnormal{k}}}
\def\rl{{\textnormal{l}}}
% rm is already a command, just don't name any random variables m
\def\rn{{\textnormal{n}}}
\def\ro{{\textnormal{o}}}
\def\rp{{\textnormal{p}}}
\def\rq{{\textnormal{q}}}
\def\rr{{\textnormal{r}}}
\def\rs{{\textnormal{s}}}
\def\rt{{\textnormal{t}}}
\def\ru{{\textnormal{u}}}
\def\rv{{\textnormal{v}}}
\def\rw{{\textnormal{w}}}
\def\rx{{\textnormal{x}}}
\def\ry{{\textnormal{y}}}
\def\rz{{\textnormal{z}}}

% Random vectors
\def\rvepsilon{{\mathbf{\epsilon}}}
\def\rvtheta{{\mathbf{\theta}}}
\def\rva{{\mathbf{a}}}
\def\rvb{{\mathbf{b}}}
\def\rvc{{\mathbf{c}}}
\def\rvd{{\mathbf{d}}}
\def\rve{{\mathbf{e}}}
\def\rvf{{\mathbf{f}}}
\def\rvg{{\mathbf{g}}}
\def\rvh{{\mathbf{h}}}
\def\rvu{{\mathbf{i}}}
\def\rvj{{\mathbf{j}}}
\def\rvk{{\mathbf{k}}}
\def\rvl{{\mathbf{l}}}
\def\rvm{{\mathbf{m}}}
\def\rvn{{\mathbf{n}}}
\def\rvo{{\mathbf{o}}}
\def\rvp{{\mathbf{p}}}
\def\rvq{{\mathbf{q}}}
\def\rvr{{\mathbf{r}}}
\def\rvs{{\mathbf{s}}}
\def\rvt{{\mathbf{t}}}
\def\rvu{{\mathbf{u}}}
\def\rvv{{\mathbf{v}}}
\def\rvw{{\mathbf{w}}}
\def\rvx{{\mathbf{x}}}
\def\rvy{{\mathbf{y}}}
\def\rvz{{\mathbf{z}}}

% Elements of random vectors
\def\erva{{\textnormal{a}}}
\def\ervb{{\textnormal{b}}}
\def\ervc{{\textnormal{c}}}
\def\ervd{{\textnormal{d}}}
\def\erve{{\textnormal{e}}}
\def\ervf{{\textnormal{f}}}
\def\ervg{{\textnormal{g}}}
\def\ervh{{\textnormal{h}}}
\def\ervi{{\textnormal{i}}}
\def\ervj{{\textnormal{j}}}
\def\ervk{{\textnormal{k}}}
\def\ervl{{\textnormal{l}}}
\def\ervm{{\textnormal{m}}}
\def\ervn{{\textnormal{n}}}
\def\ervo{{\textnormal{o}}}
\def\ervp{{\textnormal{p}}}
\def\ervq{{\textnormal{q}}}
\def\ervr{{\textnormal{r}}}
\def\ervs{{\textnormal{s}}}
\def\ervt{{\textnormal{t}}}
\def\ervu{{\textnormal{u}}}
\def\ervv{{\textnormal{v}}}
\def\ervw{{\textnormal{w}}}
\def\ervx{{\textnormal{x}}}
\def\ervy{{\textnormal{y}}}
\def\ervz{{\textnormal{z}}}

% Random matrices
\def\rmA{{\mathbf{A}}}
\def\rmB{{\mathbf{B}}}
\def\rmC{{\mathbf{C}}}
\def\rmD{{\mathbf{D}}}
\def\rmE{{\mathbf{E}}}
\def\rmF{{\mathbf{F}}}
\def\rmG{{\mathbf{G}}}
\def\rmH{{\mathbf{H}}}
\def\rmI{{\mathbf{I}}}
\def\rmJ{{\mathbf{J}}}
\def\rmK{{\mathbf{K}}}
\def\rmL{{\mathbf{L}}}
\def\rmM{{\mathbf{M}}}
\def\rmN{{\mathbf{N}}}
\def\rmO{{\mathbf{O}}}
\def\rmP{{\mathbf{P}}}
\def\rmQ{{\mathbf{Q}}}
\def\rmR{{\mathbf{R}}}
\def\rmS{{\mathbf{S}}}
\def\rmT{{\mathbf{T}}}
\def\rmU{{\mathbf{U}}}
\def\rmV{{\mathbf{V}}}
\def\rmW{{\mathbf{W}}}
\def\rmX{{\mathbf{X}}}
\def\rmY{{\mathbf{Y}}}
\def\rmZ{{\mathbf{Z}}}

% Elements of random matrices
\def\ermA{{\textnormal{A}}}
\def\ermB{{\textnormal{B}}}
\def\ermC{{\textnormal{C}}}
\def\ermD{{\textnormal{D}}}
\def\ermE{{\textnormal{E}}}
\def\ermF{{\textnormal{F}}}
\def\ermG{{\textnormal{G}}}
\def\ermH{{\textnormal{H}}}
\def\ermI{{\textnormal{I}}}
\def\ermJ{{\textnormal{J}}}
\def\ermK{{\textnormal{K}}}
\def\ermL{{\textnormal{L}}}
\def\ermM{{\textnormal{M}}}
\def\ermN{{\textnormal{N}}}
\def\ermO{{\textnormal{O}}}
\def\ermP{{\textnormal{P}}}
\def\ermQ{{\textnormal{Q}}}
\def\ermR{{\textnormal{R}}}
\def\ermS{{\textnormal{S}}}
\def\ermT{{\textnormal{T}}}
\def\ermU{{\textnormal{U}}}
\def\ermV{{\textnormal{V}}}
\def\ermW{{\textnormal{W}}}
\def\ermX{{\textnormal{X}}}
\def\ermY{{\textnormal{Y}}}
\def\ermZ{{\textnormal{Z}}}

% Vectors
\def\vzero{{\bm{0}}}
\def\vone{{\bm{1}}}
\def\vmu{{\bm{\mu}}}
\def\vtheta{{\bm{\theta}}}
\def\va{{\bm{a}}}
\def\vb{{\bm{b}}}
\def\vc{{\bm{c}}}
\def\vd{{\bm{d}}}
\def\ve{{\bm{e}}}
\def\vf{{\bm{f}}}
\def\vg{{\bm{g}}}
\def\vh{{\bm{h}}}
\def\vi{{\bm{i}}}
\def\vj{{\bm{j}}}
\def\vk{{\bm{k}}}
\def\vl{{\bm{l}}}
\def\vm{{\bm{m}}}
\def\vn{{\bm{n}}}
\def\vo{{\bm{o}}}
\def\vp{{\bm{p}}}
\def\vq{{\bm{q}}}
\def\vr{{\bm{r}}}
\def\vs{{\bm{s}}}
\def\vt{{\bm{t}}}
\def\vu{{\bm{u}}}
\def\vv{{\bm{v}}}
\def\vw{{\bm{w}}}
\def\vx{{\bm{x}}}
\def\vy{{\bm{y}}}
\def\vz{{\bm{z}}}

% Elements of vectors
\def\evalpha{{\alpha}}
\def\evbeta{{\beta}}
\def\evepsilon{{\epsilon}}
\def\evlambda{{\lambda}}
\def\evomega{{\omega}}
\def\evmu{{\mu}}
\def\evpsi{{\psi}}
\def\evsigma{{\sigma}}
\def\evtheta{{\theta}}
\def\eva{{a}}
\def\evb{{b}}
\def\evc{{c}}
\def\evd{{d}}
\def\eve{{e}}
\def\evf{{f}}
\def\evg{{g}}
\def\evh{{h}}
\def\evi{{i}}
\def\evj{{j}}
\def\evk{{k}}
\def\evl{{l}}
\def\evm{{m}}
\def\evn{{n}}
\def\evo{{o}}
\def\evp{{p}}
\def\evq{{q}}
\def\evr{{r}}
\def\evs{{s}}
\def\evt{{t}}
\def\evu{{u}}
\def\evv{{v}}
\def\evw{{w}}
\def\evx{{x}}
\def\evy{{y}}
\def\evz{{z}}

% Matrix
\def\mA{{\bm{A}}}
\def\mB{{\bm{B}}}
\def\mC{{\bm{C}}}
\def\mD{{\bm{D}}}
\def\mE{{\bm{E}}}
\def\mF{{\bm{F}}}
\def\mG{{\bm{G}}}
\def\mH{{\bm{H}}}
\def\mI{{\bm{I}}}
\def\mJ{{\bm{J}}}
\def\mK{{\bm{K}}}
\def\mL{{\bm{L}}}
\def\mM{{\bm{M}}}
\def\mN{{\bm{N}}}
\def\mO{{\bm{O}}}
\def\mP{{\bm{P}}}
\def\mQ{{\bm{Q}}}
\def\mR{{\bm{R}}}
\def\mS{{\bm{S}}}
\def\mT{{\bm{T}}}
\def\mU{{\bm{U}}}
\def\mV{{\bm{V}}}
\def\mW{{\bm{W}}}
\def\mX{{\bm{X}}}
\def\mY{{\bm{Y}}}
\def\mZ{{\bm{Z}}}
\def\mBeta{{\bm{\beta}}}
\def\mPhi{{\bm{\Phi}}}
\def\mLambda{{\bm{\Lambda}}}
\def\mSigma{{\bm{\Sigma}}}

% Tensor
\DeclareMathAlphabet{\mathsfit}{\encodingdefault}{\sfdefault}{m}{sl}
\SetMathAlphabet{\mathsfit}{bold}{\encodingdefault}{\sfdefault}{bx}{n}
\newcommand{\tens}[1]{\bm{\mathsfit{#1}}}
\def\tA{{\tens{A}}}
\def\tB{{\tens{B}}}
\def\tC{{\tens{C}}}
\def\tD{{\tens{D}}}
\def\tE{{\tens{E}}}
\def\tF{{\tens{F}}}
\def\tG{{\tens{G}}}
\def\tH{{\tens{H}}}
\def\tI{{\tens{I}}}
\def\tJ{{\tens{J}}}
\def\tK{{\tens{K}}}
\def\tL{{\tens{L}}}
\def\tM{{\tens{M}}}
\def\tN{{\tens{N}}}
\def\tO{{\tens{O}}}
\def\tP{{\tens{P}}}
\def\tQ{{\tens{Q}}}
\def\tR{{\tens{R}}}
\def\tS{{\tens{S}}}
\def\tT{{\tens{T}}}
\def\tU{{\tens{U}}}
\def\tV{{\tens{V}}}
\def\tW{{\tens{W}}}
\def\tX{{\tens{X}}}
\def\tY{{\tens{Y}}}
\def\tZ{{\tens{Z}}}


% Graph
\def\gA{{\mathcal{A}}}
\def\gB{{\mathcal{B}}}
\def\gC{{\mathcal{C}}}
\def\gD{{\mathcal{D}}}
\def\gE{{\mathcal{E}}}
\def\gF{{\mathcal{F}}}
\def\gG{{\mathcal{G}}}
\def\gH{{\mathcal{H}}}
\def\gI{{\mathcal{I}}}
\def\gJ{{\mathcal{J}}}
\def\gK{{\mathcal{K}}}
\def\gL{{\mathcal{L}}}
\def\gM{{\mathcal{M}}}
\def\gN{{\mathcal{N}}}
\def\gO{{\mathcal{O}}}
\def\gP{{\mathcal{P}}}
\def\gQ{{\mathcal{Q}}}
\def\gR{{\mathcal{R}}}
\def\gS{{\mathcal{S}}}
\def\gT{{\mathcal{T}}}
\def\gU{{\mathcal{U}}}
\def\gV{{\mathcal{V}}}
\def\gW{{\mathcal{W}}}
\def\gX{{\mathcal{X}}}
\def\gY{{\mathcal{Y}}}
\def\gZ{{\mathcal{Z}}}

% Sets
\def\sA{{\mathbb{A}}}
\def\sB{{\mathbb{B}}}
\def\sC{{\mathbb{C}}}
\def\sD{{\mathbb{D}}}
% Don't use a set called E, because this would be the same as our symbol
% for expectation.
\def\sF{{\mathbb{F}}}
\def\sG{{\mathbb{G}}}
\def\sH{{\mathbb{H}}}
\def\sI{{\mathbb{I}}}
\def\sJ{{\mathbb{J}}}
\def\sK{{\mathbb{K}}}
\def\sL{{\mathbb{L}}}
\def\sM{{\mathbb{M}}}
\def\sN{{\mathbb{N}}}
\def\sO{{\mathbb{O}}}
\def\sP{{\mathbb{P}}}
\def\sQ{{\mathbb{Q}}}
\def\sR{{\mathbb{R}}}
\def\sS{{\mathbb{S}}}
\def\sT{{\mathbb{T}}}
\def\sU{{\mathbb{U}}}
\def\sV{{\mathbb{V}}}
\def\sW{{\mathbb{W}}}
\def\sX{{\mathbb{X}}}
\def\sY{{\mathbb{Y}}}
\def\sZ{{\mathbb{Z}}}

% Entries of a matrix
\def\emLambda{{\Lambda}}
\def\emA{{A}}
\def\emB{{B}}
\def\emC{{C}}
\def\emD{{D}}
\def\emE{{E}}
\def\emF{{F}}
\def\emG{{G}}
\def\emH{{H}}
\def\emI{{I}}
\def\emJ{{J}}
\def\emK{{K}}
\def\emL{{L}}
\def\emM{{M}}
\def\emN{{N}}
\def\emO{{O}}
\def\emP{{P}}
\def\emQ{{Q}}
\def\emR{{R}}
\def\emS{{S}}
\def\emT{{T}}
\def\emU{{U}}
\def\emV{{V}}
\def\emW{{W}}
\def\emX{{X}}
\def\emY{{Y}}
\def\emZ{{Z}}
\def\emSigma{{\Sigma}}

% entries of a tensor
% Same font as tensor, without \bm wrapper
\newcommand{\etens}[1]{\mathsfit{#1}}
\def\etLambda{{\etens{\Lambda}}}
\def\etA{{\etens{A}}}
\def\etB{{\etens{B}}}
\def\etC{{\etens{C}}}
\def\etD{{\etens{D}}}
\def\etE{{\etens{E}}}
\def\etF{{\etens{F}}}
\def\etG{{\etens{G}}}
\def\etH{{\etens{H}}}
\def\etI{{\etens{I}}}
\def\etJ{{\etens{J}}}
\def\etK{{\etens{K}}}
\def\etL{{\etens{L}}}
\def\etM{{\etens{M}}}
\def\etN{{\etens{N}}}
\def\etO{{\etens{O}}}
\def\etP{{\etens{P}}}
\def\etQ{{\etens{Q}}}
\def\etR{{\etens{R}}}
\def\etS{{\etens{S}}}
\def\etT{{\etens{T}}}
\def\etU{{\etens{U}}}
\def\etV{{\etens{V}}}
\def\etW{{\etens{W}}}
\def\etX{{\etens{X}}}
\def\etY{{\etens{Y}}}
\def\etZ{{\etens{Z}}}

% The true underlying data generating distribution
\newcommand{\pdata}{p_{\rm{data}}}
% The empirical distribution defined by the training set
\newcommand{\ptrain}{\hat{p}_{\rm{data}}}
\newcommand{\Ptrain}{\hat{P}_{\rm{data}}}
% The model distribution
\newcommand{\pmodel}{p_{\rm{model}}}
\newcommand{\Pmodel}{P_{\rm{model}}}
\newcommand{\ptildemodel}{\tilde{p}_{\rm{model}}}
% Stochastic autoencoder distributions
\newcommand{\pencode}{p_{\rm{encoder}}}
\newcommand{\pdecode}{p_{\rm{decoder}}}
\newcommand{\precons}{p_{\rm{reconstruct}}}

\newcommand{\laplace}{\mathrm{Laplace}} % Laplace distribution

\newcommand{\E}{\mathbb{E}}
\newcommand{\Ls}{\mathcal{L}}
\newcommand{\R}{\mathbb{R}}
\newcommand{\emp}{\tilde{p}}
\newcommand{\lr}{\alpha}
\newcommand{\reg}{\lambda}
\newcommand{\rect}{\mathrm{rectifier}}
\newcommand{\softmax}{\mathrm{softmax}}
\newcommand{\sigmoid}{\sigma}
\newcommand{\softplus}{\zeta}
\newcommand{\KL}{D_{\mathrm{KL}}}
\newcommand{\Var}{\mathrm{Var}}
\newcommand{\standarderror}{\mathrm{SE}}
\newcommand{\Cov}{\mathrm{Cov}}
% Wolfram Mathworld says $L^2$ is for function spaces and $\ell^2$ is for vectors
% But then they seem to use $L^2$ for vectors throughout the site, and so does
% wikipedia.
\newcommand{\normlzero}{L^0}
\newcommand{\normlone}{L^1}
\newcommand{\normltwo}{L^2}
\newcommand{\normlp}{L^p}
\newcommand{\normmax}{L^\infty}

\newcommand{\parents}{Pa} % See usage in notation.tex. Chosen to match Daphne's book.

\DeclareMathOperator*{\argmax}{arg\,max}
\DeclareMathOperator*{\argmin}{arg\,min}

\DeclareMathOperator{\sign}{sign}
\DeclareMathOperator{\Tr}{Tr}
\let\ab\allowbreak

\newcommand{\cX}{\mathcal{X}}
\newcommand{\cD}{\mathcal{D}}

% Title page information
\title[Information-Theoretic Approaches to ML Reliability]{Information-Theoretic Approaches to Out-of-Distribution Detection and Hallucination Detection in Machine Learning Systems}
\subtitle{PhD Dissertation Proposal Defense}
\author{Your Name}
\institute{Your University}
\date{\today}

% Remove navigation symbols
\setbeamertemplate{navigation symbols}{}

% Custom footline with section indicator
\setbeamertemplate{footline}{
  \leavevmode%
  \hbox{%
  \begin{beamercolorbox}[wd=.5\paperwidth,ht=2.25ex,dp=1ex,left]{author in head/foot}%
    \usebeamerfont{author in head/foot}\hspace*{2ex}\insertsectionhead
  \end{beamercolorbox}%
  \begin{beamercolorbox}[wd=.5\paperwidth,ht=2.25ex,dp=1ex,right]{title in head/foot}%
    \usebeamerfont{title in head/foot}\insertframenumber{} / \inserttotalframenumber\hspace*{2ex}
  \end{beamercolorbox}}%
  \vskip0pt%
}

\begin{document}

% 1. Title slide
\begin{frame}
\titlepage
\end{frame}

% SECTION 1: INTRODUCTION (4 slides, ~5 minutes)
\section{Introduction}

% 2. The Reliability Crisis in Modern ML
\begin{frame}{The Reliability Crisis in Modern ML}
\begin{itemize}
    \item \textbf{Safety-Critical Deployments}: ML systems in healthcare, autonomous driving, and financial services
    \item \textbf{Two Fundamental Failures}:
    \begin{itemize}
        \item \textcolor{red}{Overconfident predictions} on unfamiliar inputs (OOD detection)
        \item \textcolor{red}{Plausible but false content} generation (hallucination detection)
    \end{itemize}
    \item \textbf{Real-World Consequences}:
    \begin{itemize}
        \item Medical imaging models misclassifying rare conditions
        \item Autonomous vehicles failing on novel scenarios
        \item AI assistants providing incorrect medical/legal advice
    \end{itemize}
    \item \textbf{Current Gap}: Lack of principled theoretical frameworks for reliability
\end{itemize}
\end{frame}

% 3. Information Theory Foundations
\begin{frame}{Information Theory Foundations}
\begin{itemize}
    \item \textbf{Mutual Information}: $I(\mathbf{X}; \mathbf{Y}) = H(\mathbf{X}) - H(\mathbf{X}|\mathbf{Y})$
    \begin{itemize}
        \item Quantifies shared information between variables
        \item Measures reduction in uncertainty about $\mathbf{X}$ given $\mathbf{Y}$
    \end{itemize}
    \item \textbf{Information Bottleneck Principle}:
    $$\mathcal{L}_{IB} = I(\mathbf{Z}; \mathbf{Y}) - \beta I(\mathbf{Z}; \mathbf{X})$$
    \begin{itemize}
        \item Compress toward minimal sufficient statistics
        \item Discard "irrelevant" information during learning
    \end{itemize}
    \item \textbf{Why Information Theory for ML Reliability?}
    \begin{itemize}
        \item Provides quantifiable, objective measures of uncertainty
        \item Unifies OOD detection and hallucination detection under common framework
    \end{itemize}
\end{itemize}
\end{frame}

% 4. Presentation Roadmap
\begin{frame}{Presentation Roadmap}
\begin{itemize}
    \item \textbf{Three Interconnected Contributions}:
    \begin{enumerate}
        \item \textcolor{blue}{Label Blindness}: When unlabeled OOD detection ignores critical information
        \item \textcolor{green}{Domain Feature Collapse}: Why single-domain models fail at OOD detection
        \item \textcolor{red}{Hallucination Detection}: Information-theoretic framework for LLM reliability
    \end{enumerate}
    \item \textbf{Presentation Structure}:
    \begin{itemize}
        \item High-level overview of all three contributions
        \item Deep technical dive into each contribution
        \item Research timeline and expected impact
    \end{itemize}
    \item \textbf{Unifying Theme}: Information theory as principled framework for AI safety
\end{itemize}
\end{frame}

% SECTION 2: HIGH-LEVEL OVERVIEW OF THREE CONTRIBUTIONS (3 slides, ~6 minutes)
\section{Research Overview}

% 5. Label Blindness Overview
\begin{frame}{Contribution 1: Label Blindness in Unlabeled OOD Detection}
\begin{itemize}
    \item \textbf{Core Problem}: Unlabeled OOD detection methods ignore critical label information
    \begin{itemize}
        \item When $I(\mathbf{z}_{unsup}; \mathbf{y}) = 0$ (feature independence from labels)
        \item Guaranteed failure when unsupervised features $\perp$ supervised features
    \end{itemize}
    \item \textbf{Novel Insight}: \textcolor{blue}{Adjacent OOD} evaluation paradigm
    \begin{itemize}
        \item Example: Dog breeds dataset - 80\% breeds as ID, 20\% breeds as OOD
        \item Reveals systematic failures hidden by traditional distant OOD benchmarks
    \end{itemize}
    \item \textbf{Theoretical Contribution}: Label Blindness Theorem
    \begin{itemize}
        \item Formal proof of when and why unlabeled methods fail
        \item Information-theoretic conditions for detection success/failure
    \end{itemize}
    \item \textbf{Practical Impact}: Guides method selection and hybrid approaches
\end{itemize}
\end{frame}

% 6. Domain Feature Collapse Overview
\begin{frame}{Contribution 2: Domain Feature Collapse}
\begin{itemize}
    \item \textbf{Phenomenon}: Single-domain training discards domain-specific features
    \begin{itemize}
        \item $I(\mathbf{x}_d; \mathbf{z}) = 0$ for learned representations $\mathbf{z}$
        \item Example: X-ray model confidently classifying MRI scans
    \end{itemize}
    \item \textbf{Theoretical Foundation}: Information Bottleneck drives inevitable collapse
    \begin{itemize}
        \item $\mathcal{L}_{IB} = I(\mathbf{Z}; \mathbf{Y}) - \beta I(\mathbf{Z}; \mathbf{X})$
        \item Domain features $\mathbf{x}_d$ discarded when $I(\mathbf{x}_d; \mathbf{Y}) = 0$
        \item Mathematical proof of collapse under supervised learning
    \end{itemize}
    \item \textbf{Solution}: Two-stage domain filtering framework
    \begin{itemize}
        \item Stage 1: Domain-level detection (preserve $\mathbf{x}_d$ during training)
        \item Stage 2: Class-level detection within correct domain
    \end{itemize}
    \item \textbf{Impact}: First formal characterization + practical mitigation strategy
\end{itemize}
\end{frame}

% 7. Hallucination Detection Overview
\begin{frame}{Contribution 3: Information-Theoretic Hallucination Detection}
\begin{itemize}
    \item \textbf{Central Hypothesis}: Hallucinations arise from insufficient mutual information
    \begin{itemize}
        \item $I(\mathbf{x}; \mathbf{y}) < \tau_{critical}$ between queries and responses
        \item Layer-wise information degradation in transformer architectures
    \end{itemize}
    \item \textbf{Novel Method}: Contrastive mutual information estimation
    \begin{itemize}
        \item Real-time detection without external knowledge bases
        \item Scalable to large language models (GPT, BERT, T5, Mamba)
        \item Question-answer consistency across transformer layers
    \end{itemize}
    \item \textbf{System Architecture}: Two-stage detection framework
    \begin{itemize}
        \item Primary model: Standard transformer inference
        \item Secondary analysis: Contrastive MI estimation
    \end{itemize}
    \item \textbf{Validation}: Natural Questions, TriviaQA, HaluEval, TruthfulQA, HalluLens
\end{itemize}
\end{frame}

% SECTION 3: DETAILED TECHNICAL CONTRIBUTIONS (18 slides, ~30 minutes)

% 3a. Label Blindness Deep Dive (6 slides)
\section{Label Blindness Deep Dive}

% 8. Formal Problem Definition
\begin{frame}{Formal Problem Definition}
\begin{itemize}
    \item \textbf{Out-of-Distribution Detection Task}:
    \begin{itemize}
        \item Given: Training data $\mathcal{D}_{train} = \{(\mathbf{x}_i, y_i)\}_{i=1}^n$ from distribution $P_{ID}$
        \item Goal: Detect test samples $\mathbf{x}_{test} \sim P_{OOD}$ where $P_{OOD} \neq P_{ID}$
    \end{itemize}
    \item \textbf{Unlabeled vs. Supervised Methods}:
    \begin{itemize}
        \item Unlabeled: Use only $\{\mathbf{x}_i\}$ (ignore labels $\{y_i\}$)
        \item Supervised: Use full training data $\{(\mathbf{x}_i, y_i)\}$
    \end{itemize}
    \item \textbf{Label Blindness Definition}:
    \begin{itemize}
        \item Unlabeled method fails when $I(\mathbf{z}_{unsup}; \mathbf{y}) = 0$
        \item Where $\mathbf{z}_{unsup}$ are features learned without supervision
    \end{itemize}
    \item \textbf{Research Question}: When do unlabeled methods systematically fail?
\end{itemize}
\end{frame}

% 9. Information-Theoretic Analysis
\begin{frame}{Information-Theoretic Analysis}
\begin{itemize}
    \item \textbf{Information Bottleneck in Unsupervised Learning}:
    \begin{align}
        \mathcal{L}_{unsup} = I(\mathbf{Z}; \mathbf{X}) - \beta I(\mathbf{Z}; \mathbf{Y})
    \end{align}
    \item \textbf{Bottleneck Compression Effect}:
    \begin{itemize}
        \item Unsupervised methods minimize $I(\mathbf{Z}; \mathbf{X})$ without label guidance
        \item Compression discards features that correlate with labels $\mathbf{Y}$
        \item Result: $I(\mathbf{z}_{unsup}; \mathbf{y}) \rightarrow 0$ as compression increases
    \end{itemize}
    \item \textbf{Label Blindness Theorem}: When bottleneck compression removes label-relevant features:
    \begin{align}
        I(\mathbf{z}_{unsup}; \mathbf{y}) = 0 \Rightarrow \text{AUC}_{f_{unsup}} \leq 0.5 + \epsilon
    \end{align}
    \item \textbf{Critical Insight}: Unsupervised compression inherently conflicts with label preservation
\end{itemize}
\end{frame}

% 10. Adjacent OOD Evaluation Paradigm
\begin{frame}{Adjacent OOD Evaluation Paradigm}
\begin{itemize}
    \item \textbf{Traditional OOD Benchmarks} (hide label blindness):
    \begin{itemize}
        \item CIFAR-10 (ID) vs. SVHN (OOD) - different domains, easy to distinguish
        \item ImageNet vs. Textures - unsupervised features sufficient
    \end{itemize}
    \item \textbf{Adjacent OOD Protocol}:
    \begin{itemize}
        \item Split single dataset: 80\% classes as ID, 20\% classes as OOD
        \item Examples: Dog breeds, Bird species, Fine-grained categories
        \item Forces reliance on label-dependent features
    \end{itemize}
    \item \textbf{Key Insight}: Adjacent OOD reveals when $I(\mathbf{z}_{unsup}; \mathbf{y}) \approx 0$
    \item \textbf{Experimental Validation}:
    \begin{itemize}
        \item Unlabeled methods: 50-60\% AUC (random performance)
        \item Supervised methods: 80-90\% AUC (strong performance)
    \end{itemize}
\end{itemize}
\end{frame}

% 11. Empirical Validation Results
\begin{frame}{Empirical Validation Results}
\begin{itemize}
    \item \textbf{Adjacent OOD Benchmark}:
    \begin{itemize}
        \item Faces, Cars, Food datasets (1/3 classes held out as OOD)
        \item Repeated 5 times with different random seeds
    \end{itemize}
    \item \textbf{Methods Compared}:
    \begin{itemize}
        \item Unlabeled: SimCLR KNN, SimCLR SSD
        \item Supervised: MSP (Maximum Softmax Probability)
    \end{itemize}
    \item \textbf{Results Summary} (AUROC scores):
    \begin{itemize}
        \item \textbf{Faces}: Supervised MSP 70.8±0.3, SimCLR KNN 52.0±4.2
        \item \textbf{Cars}: Supervised MSP 69.2±0.9, SimCLR KNN 52.5±0.4
        \item \textbf{Food}: Supervised MSP 78.8±1.2, SimCLR KNN 61.1±2.8
    \end{itemize}
    \item \textbf{Key Finding}: Unlabeled methods perform near-random ($\approx$50\% AUROC)
\end{itemize}
\end{frame}

% 12. Hybrid Approach Solutions
\begin{frame}{Hybrid Approach Solutions}
\begin{itemize}
    \item \textbf{Motivation}: Combine strengths of both approaches
    \begin{itemize}
        \item Unlabeled: Good for distant OOD (domain shift)
        \item Supervised: Essential for adjacent OOD (within-domain)
    \end{itemize}
    \item \textbf{Hybrid Architecture}:
    \begin{align}
        \text{Score}_{hybrid} = \alpha \cdot \text{Score}_{unsup} + (1-\alpha) \cdot \text{Score}_{sup}
    \end{align}
    \item \textbf{Adaptive Weighting Strategy}:
    \begin{itemize}
        \item Estimate $I(\mathbf{z}_{unsup}; \mathbf{y})$ during training
        \item High MI $\Rightarrow$ increase $\alpha$ (trust unsupervised)
        \item Low MI $\Rightarrow$ decrease $\alpha$ (trust supervised)
    \end{itemize}
    \item \textbf{Performance}: Achieves best of both worlds across OOD types
\end{itemize}
\end{frame}

% 13. Label Blindness: Key Takeaways
\begin{frame}{Label Blindness: Key Takeaways}
\begin{itemize}
    \item \textbf{Theoretical Contribution}:
    \begin{itemize}
        \item First formal characterization of when unlabeled OOD detection fails
        \item Information-theoretic conditions: $I(\mathbf{z}_{unsup}; \mathbf{y}) = 0$
        \item Rigorous proof connecting mutual information to detection performance
    \end{itemize}
    \item \textbf{Methodological Innovation}:
    \begin{itemize}
        \item Adjacent OOD evaluation paradigm reveals hidden failures
        \item Exposes limitations of current benchmarking practices
    \end{itemize}
    \item \textbf{Practical Impact}:
    \begin{itemize}
        \item Guides method selection based on OOD type
        \item Hybrid approaches for robust detection across scenarios
    \end{itemize}
    \item \textbf{Future Directions}: Extend to other unsupervised learning tasks
\end{itemize}
\end{frame}

% 3b. Domain Feature Collapse Deep Dive (6 slides)
\section{Domain Feature Collapse Deep Dive}

% 14. Mathematical Formalization
\begin{frame}{Mathematical Formalization}
\begin{itemize}
    \item \textbf{Single-Domain Dataset Definition}:
    \begin{itemize}
        \item Input $\mathbf{x} = [\mathbf{x}_d, \mathbf{x}_y]$ where $\mathbf{x}_d$ are domain features, $\mathbf{x}_y$ are class features
        \item Domain features: imaging modality, sensor type, capture conditions
        \item All samples share same domain: $f_d(\mathbf{x}_d) = d_1$ (constant)
    \end{itemize}
    \item \textbf{Information Bottleneck Objective}:
    \begin{align}
        \mathcal{L}_{IB} = I(\mathbf{Z}; \mathbf{Y}) - \beta I(\mathbf{Z}; \mathbf{X})
    \end{align}
    \item \textbf{Domain Feature Collapse Theorem}: For single-domain training:
    \begin{align}
        I(\mathbf{x}_d; \mathbf{y}) = 0 \Rightarrow I(\mathbf{x}_d; \mathbf{z}) = 0
    \end{align}
    \item \textbf{Consequence}: Learned representations $\mathbf{z}$ contain no domain information
\end{itemize}
\end{frame}

% 15. Theoretical Analysis of Collapse
\begin{frame}{Theoretical Analysis of Collapse}
\begin{itemize}
    \item \textbf{Why Collapse Occurs}:
    \begin{itemize}
        \item Domain features $\mathbf{x}_d$ are independent of labels: $I(\mathbf{x}_d; \mathbf{y}) = 0$
        \item Including $\mathbf{x}_d$ in $\mathbf{z}$ increases complexity without improving prediction
        \item Bottleneck compression discards "irrelevant" domain information
    \end{itemize}
    \item \textbf{Formal Proof Sketch}:
    \begin{itemize}
        \item Optimal $\mathbf{z}$ minimizes $\mathcal{L}_{IB} = I(\mathbf{Z}; \mathbf{Y}) - \beta I(\mathbf{Z}; \mathbf{X})$
        \item Since $I(\mathbf{x}_d; \mathbf{y}) = 0$, domain features only contribute to complexity term
        \item Therefore: $I(\mathbf{x}_d; \mathbf{z}) = 0$ in optimal representation
    \end{itemize}
    \item \textbf{Real-World Implications}: Even partial compression leads to unsafe OOD detection
    \item \textbf{Fano's Inequality}: Small $I(\mathbf{x}_d; \mathbf{z})$ still causes unreliable detection
\end{itemize}
\end{frame}

% 16. Empirical Demonstration
\begin{frame}{Empirical Demonstration}
\begin{itemize}
    \item \textbf{Domain Bench}: 11 single-domain datasets
    \begin{itemize}
        \item Medical: Tissue (kidney cortex microscopy)
        \item Agriculture: Plant (leaf disease classification)
        \item Geology: Rock (mineral classification)
        \item Waste Management: Garbage (material classification)
        \item Fitness: Yoga (pose classification)
    \end{itemize}
    \item \textbf{Experimental Setup}:
    \begin{itemize}
        \item In-domain OOD: Adjacent OOD (25\% classes held out)
        \item Out-of-domain OOD: MNIST, SVHN, Textures, Places365, CIFAR-10/100
    \end{itemize}
    \item \textbf{Key Finding}: All current SOTA methods perform worse on certain out-of-domain sets vs. their in-domain OOD performance
    \item \textbf{Evidence}: FPR@95 increases from $<$10\% (in-domain) to $>$40\% (out-of-domain)
\end{itemize}
\end{frame}

% 17. Domain Filtering Methodology
\begin{frame}{Domain Filtering Methodology}
\begin{itemize}
    \item \textbf{Two-Stage Detection Framework}:
    \begin{itemize}
        \item \textbf{Stage 1}: Domain filtering - Is sample in-domain?
        \item \textbf{Stage 2}: OOD detection - Is in-domain sample in-distribution?
    \end{itemize}
    \item \textbf{Domain Filter Implementation}:
    \begin{itemize}
        \item Pretrained DinoV2 ViT-S/14 for domain-aware features
        \item KNN distance at 99th percentile threshold ($K=50$)
        \item Preserves domain-specific information during training
    \end{itemize}
    \item \textbf{Key Assumption}: No in-distribution samples are out-of-domain
    \begin{itemize}
        \item Consistent with single-domain dataset definition
        \item Allows clean separation of domain vs. class detection
    \end{itemize}
    \item \textbf{Integration}: Compatible with any existing OOD detection method
\end{itemize}
\end{frame}

% 18. Implementation and Validation
\begin{frame}{Implementation and Validation}
\begin{itemize}
    \item \textbf{Experimental Results}:
    \begin{itemize}
        \item \textbf{Domain filtering effectiveness}: FPR@95 reduced from $>$40\% to $<$5\%
        \item \textbf{Consistent improvement}: Works across all 11 single-domain datasets
        \item \textbf{Empirically validated}: Works with KNN, ReAct, and MDS methods
    \end{itemize}
    \item \textbf{Performance Metrics}:
    \begin{itemize}
        \item Out-of-domain OOD: Substantial FPR@95 reduction (8x improvement)
        \item In-domain OOD: Minimal performance impact (maintains baseline)
        \item AUROC improvements: 15-25 percentage points on out-of-domain
    \end{itemize}
    \item \textbf{Validation Across Domains}:
    \begin{itemize}
        \item Medical imaging, agriculture, geology, waste management
        \item Confirms theoretical predictions empirically
    \end{itemize}
\end{itemize}
\end{frame}

% 19. Domain Collapse: Key Takeaways
\begin{frame}{Domain Collapse: Key Takeaways}
\begin{itemize}
    \item \textbf{Theoretical Breakthrough}:
    \begin{itemize}
        \item First formal proof of domain feature collapse using information bottleneck theory
        \item Explains why single-domain training creates dangerous OOD detection blind spots
        \item Connects supervised learning objectives to systematic safety failures
    \end{itemize}
    \item \textbf{Practical Solution}:
    \begin{itemize}
        \item Domain filtering: Simple, effective, and method-agnostic approach
        \item Two-stage framework preserves both domain and class detection capabilities
        \item 8x improvement in out-of-domain OOD detection performance
    \end{itemize}
    \item \textbf{Broader Impact}:
    \begin{itemize}
        \item Domain Bench: New benchmark for single-domain OOD evaluation
        \item Safety implications for medical imaging, autonomous systems
        \item Guides deployment decisions in safety-critical applications
    \end{itemize}
\end{itemize}
\end{frame}

% 3c. Hallucination Detection Deep Dive (6 slides)
\section{Hallucination Detection Deep Dive}

% 20. Information-Theoretic Hallucination Framework
\begin{frame}{Information-Theoretic Hallucination Framework}
\begin{itemize}
    \item \textbf{Central Hypothesis}: Hallucinations arise from information degradation
    \begin{itemize}
        \item $I(\mathbf{x}; \mathbf{y}) < \tau_{critical}$ between input queries and generated responses
        \item Layer-wise information loss in transformer architectures
        \item Critical threshold where reliable generation becomes impossible
    \end{itemize}
    \item \textbf{Information Flow Analysis}:
    \begin{itemize}
        \item Track $I(\mathbf{x}; \mathbf{z}_l)$ across transformer layers $l$
        \item Identify bottleneck layers where information degrades
        \item Attention mechanism role in preserving/destroying information
    \end{itemize}
    \item \textbf{Theoretical Foundation}: Information Bottleneck Principle
    \item \textbf{Advantage}: No external knowledge bases required for detection
\end{itemize}
\end{frame}

% 21. Contrastive MI Estimation Method
\begin{frame}{Contrastive MI Estimation Method}
\begin{itemize}
    \item \textbf{Novel Approach}: Contrastive learning for MI estimation
    \begin{itemize}
        \item Learn projections $f_i: \mathbf{z}_{l_i} \rightarrow \mathbb{R}^d$ and $f_j: \mathbf{z}_{l_j} \rightarrow \mathbb{R}^d$
        \item Maximize similarity for same QA pairs across layers
        \item Minimize similarity for different QA pairs
    \end{itemize}
    \item \textbf{Contrastive Objective}:
    \begin{align}
        \mathcal{L} = -\log \frac{\exp(\text{sim}(f_i(\mathbf{z}_{l_i}), f_j(\mathbf{z}_{l_j})) / \tau)}{\sum_{k=1}^{N} \exp(\text{sim}(f_i(\mathbf{z}_{l_i}), f_j(\mathbf{z}_{l_j}^{(k)})) / \tau)}
    \end{align}
    \item \textbf{MI Estimation}: $\hat{I}(\mathbf{z}_{l_i}; \mathbf{z}_{l_j})$ from learned representations
    \item \textbf{Advantages}: Task-specific, scalable, differentiable
\end{itemize}
\end{frame}

% 22. System Architecture Details
\begin{frame}{System Architecture Details}
\begin{itemize}
    \item \textbf{Two-Stage Detection Framework}:
    \begin{itemize}
        \item \textbf{Stage 1}: Primary LM generates responses + extracts layer embeddings
        \item \textbf{Stage 2}: Secondary analysis model estimates MI between layers
        \item Real-time detection during inference
    \end{itemize}
    \item \textbf{Training Process}:
    \begin{itemize}
        \item Use QA pairs with hallucination labels for contrastive learning
        \item Learn to distinguish faithful vs. hallucinated responses
        \item Optimize projection functions for MI estimation
    \end{itemize}
    \item \textbf{Detection Mechanism}: $\hat{I}(\mathbf{x}; \mathbf{y}) < \tau_{critical}$ triggers hallucination alert
    \item \textbf{Cross-Architecture Compatibility}: GPT, BERT, T5, Mamba
\end{itemize}
\end{frame}

% 23. Experimental Design and Datasets
\begin{frame}{Experimental Design and Datasets}
\begin{itemize}
    \item \textbf{Validation Strategy}:
    \begin{itemize}
        \item \textbf{Synthetic datasets}: Ground truth MI for method validation
        \item \textbf{Real-world benchmarks}: HaluEval, TruthfulQA, FEVER, HalluLens
        \item Cross-method consistency analysis (MINE, InfoNCE, kernel-based)
    \end{itemize}
    \item \textbf{Evaluation Metrics}:
    \begin{itemize}
        \item MI estimation accuracy vs. ground truth
        \item Hallucination detection: AUROC, precision, recall, F1
        \item Bias-variance decomposition of MI estimates
    \end{itemize}
    \item \textbf{Model Coverage}: GPT-3.5/4, BERT, T5, LLaMA, Mamba architectures
    \item \textbf{Ablation Studies}: Projection architecture, temperature, negative sampling
\end{itemize}
\end{frame}

% 24. Results and Performance Analysis
\begin{frame}{Results and Performance Analysis}
\begin{itemize}
    \item \textbf{MI Estimation Performance}:
    \begin{itemize}
        \item Superior accuracy on QA-specific tasks vs. general MI methods
        \item Computational efficiency: 10-100x faster than MINE
        \item Robust performance across different model scales and architectures
    \end{itemize}
    \item \textbf{Hallucination Detection Results}:
    \begin{itemize}
        \item Target: $>$85\% AUROC on major benchmarks (HaluEval, TruthfulQA)
        \item Real-time detection with $<$50ms latency overhead
        \item Cross-architecture generalization without retraining
    \end{itemize}
    \item \textbf{Information Flow Insights}:
    \begin{itemize}
        \item Identify critical layers where hallucinations emerge
        \item Quantify attention mechanism role in information preservation
    \end{itemize}
\end{itemize}
\end{frame}

% 25. Hallucination Detection: Key Takeaways
\begin{frame}{Hallucination Detection: Key Takeaways}
\begin{itemize}
    \item \textbf{Theoretical Innovation}:
    \begin{itemize}
        \item First information-theoretic framework for hallucination detection
        \item Novel contrastive MI estimation method for transformer architectures
        \item Principled connection between information flow and generation reliability
    \end{itemize}
    \item \textbf{Practical Advantages}:
    \begin{itemize}
        \item Real-time detection without external knowledge bases
        \item Cross-architecture compatibility (GPT, BERT, T5, Mamba)
        \item Scalable to large language models with minimal overhead
    \end{itemize}
    \item \textbf{Research Impact}:
    \begin{itemize}
        \item Opens new research directions in information-theoretic AI safety
        \item Enables targeted interventions at critical transformer layers
        \item Foundation for next-generation trustworthy AI systems
    \end{itemize}
\end{itemize}
\end{frame}

% SECTION 4: RESEARCH TIMELINE AND CONCLUSION (3 slides, ~6 minutes)
\section{Timeline \& Conclusion}

% 26. Research Timeline
\begin{frame}{Research Timeline}
\begin{itemize}
    \item \textbf{Phase 1: Foundation \& Method Development (Months 1-4)}:
    \begin{itemize}
        \item Theoretical framework refinement and prototype analysis
        \item Implementation optimization and baseline comparisons
        \item Infrastructure setup for large-scale experiments
    \end{itemize}
    \item \textbf{Phase 2: Large-Scale Validation (Months 5-8)}:
    \begin{itemize}
        \item Foundation model integration (GPT, BERT, T5, Mamba)
        \item MI-hallucination correlation studies and detection system development
        \item Cross-architecture validation and performance benchmarking
    \end{itemize}
    \item \textbf{Phase 3: Applications \& Deployment (Months 9-12)}:
    \begin{itemize}
        \item Domain-specific applications and intervention strategies
        \item Comprehensive evaluation and open-source implementation
        \item Research dissemination and community adoption
    \end{itemize}
\end{itemize}
\end{frame}

% 27. Expected Impact and Significance
\begin{frame}{Expected Impact and Significance}
\begin{itemize}
    \item \textbf{Theoretical Contributions}:
    \begin{itemize}
        \item First comprehensive information-theoretic framework for AI safety
        \item Novel understanding of failure modes in OOD detection and hallucination
        \item Principled connection between information theory and model reliability
    \end{itemize}
    \item \textbf{Practical Applications}:
    \begin{itemize}
        \item Real-time detection systems for safety-critical deployments
        \item Cross-architecture compatibility enabling broad adoption
        \item Open-source tools for community use and further research
    \end{itemize}
    \item \textbf{Broader Impact}:
    \begin{itemize}
        \item Enhanced trustworthiness of AI systems in healthcare, finance, autonomous vehicles
        \item New research directions in information-theoretic AI safety
        \item Foundation for next-generation reliable machine learning systems
    \end{itemize}
\end{itemize}
\end{frame}

% 28. Questions and Discussion
\begin{frame}{Questions and Discussion}
\begin{center}
\Huge Thank you!

\vspace{2cm}

\Large Questions?
\end{center}
\end{frame}

\end{document}

